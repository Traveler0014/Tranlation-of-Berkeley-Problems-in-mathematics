\usepackage{tabularx}
\newcolumntype{Y}{>{\centering\arraybackslash}X}
\usepackage{manfnt}
\usepackage[T1]{fontenc}
\renewcommand\rmdefault{ptm}
\usepackage[centering,paperwidth=18.5cm,paperheight=26cm,
           top=3cm,bottom=3cm,right=2.75cm,left=2.75cm]{geometry}
\usepackage[complete,amssymbols,amsbb,eufrak,nofontinfo,
             subscriptcorrection,zswash,mtpscr]{mtpro2}
\usepackage{mathtools,mathrsfs}
\usepackage{caption}

\usepackage{extarrows}
\usepackage{graphicx,pgfornament-han}

\setmainfont{Times New Roman}

\usepackage{pifont}
\usepackage{imakeidx}
\makeindex[
title={名词索引},
intoc=true,
columns=2,
columnsep=1cm,
columnseprule=true,
program=makeindex,
options={-s mkind.ist},
noautomatic=false
]
\indexsetup{
toclevel=chapter,
headers={名词索引}{名词索引},
othercode={
\renewcommand{\indexspace}{\smallskip}
}
}

\usepackage[hyperindex]{hyperref}
\hypersetup{bookmarksopen=true,bookmarksopenlevel=1,bookmarksnumbered=true,
  pdftitle={Berkeley数学问题},pdfauthor={向禹},linktoc=all,CJKbookmarks=true,unicode,
  colorlinks,linkcolor=blue,citecolor=red,urlcolor=blue,anchorcolor=green}
\usepackage[Symbol]{upgreek}
\renewcommand{\pi}{\uppi}

\DeclareSymbolFont{ugmL}{OMX}{yhex}{m}{n}
\DeclareMathAccent{\wideparen}{\mathord}{ugmL}{"F3}

\setCJKmainfont[BoldFont={方正黑体_GBK},ItalicFont={方正楷体_GBK}
,Mapping=fullwidth-stop]{方正书宋_GBK}

\defaultfontfeatures{Mapping=tex-text}
\XeTeXlinebreaklocale ”zh”
\XeTeXlinebreakskip = 0pt plus 1pt

\setCJKfamilyfont{song}{方正书宋_GBK}
\newcommand{\song}{\CJKfamily{song}}
\setCJKfamilyfont{hei}{方正黑体_GBK}
\newcommand{\hei}{\CJKfamily{hei}}
\setCJKfamilyfont{kai}{方正楷体_GBK}
\newcommand{\kai}{\CJKfamily{kai}}
\setCJKfamilyfont{fs}{方正仿宋_GBK}
\newcommand{\fs}{\CJKfamily{fs}}
\setCJKfamilyfont{fzxk}{方正行楷_GBK}
\newcommand{\fzxk}{\CJKfamily{fzxk}}
\setCJKfamilyfont{fzqt}{方正启体简体}
\newcommand{\fzqt}{\CJKfamily{fzqt}}
%\newenvironment{Proof}{\par\indent{\hei 证}\hspace{1em}}{\hfill\ding73\par}
%\newenvironment{solve}{\par\indent{\hei 解}\hspace{1em}}{\par}


\usepackage{tasks}
\settasks{
label = (\arabic*),
item-indent = 1.7em,
label-width = 0.5em,
label-offset = 1.2em,
column-sep=10pt,
label-align=left,
after-item-skip=0pt
}
\usepackage{multicol}

%\everymath{\displaystyle}

\renewcommand{\le}{\leqslant}
\renewcommand{\ge}{\geqslant}

\allowdisplaybreaks[4]


\newcommand\OR{\overrightarrow}



\usepackage{fancyhdr,tikz}
\usetikzlibrary{arrows.meta,patterns}
\renewcommand{\Re}{\operatorname{\mathrm{Re}}}
\renewcommand{\Im}{\operatorname{\mathrm{Im}}}
\newcommand{\ii}{\mathrm i}
\ctexset{punct=kaiming}
\renewcommand\thempfootnote{\ding{45}}
\newenvironment{note}{\par\CJKfamily{note}\noindent{\makebox[0pt][r]{\scriptsize\color{red!90}
\textdbend\quad}\textbf{注:}}}{\par}
\pagestyle{fancy}
\fancyhf{}
\cfoot{\thepage}
\fancyhead[LO,RE]{\href{yuxtech.github.io}{我的博客: yuxtech.github.io}}
\fancyhead[RO,LE]{\rightmark}

\counterwithin{chapter}{part}
\renewcommand\thechapter{\arabic{chapter}}
\ctexset {
  chapter = {
  beforeskip = 0pt,
  fixskip = true,
  format = \Huge\bfseries,
  nameformat = \rule{\linewidth}{1bp}\par\bigskip\hfill\label{chap\thechapter}\chapternamebox,
  number = \arabic{chapter},
  aftername = \par\medskip,
  aftertitle = \par\bigskip\nointerlineskip\rule{\linewidth}{2bp}\par
},
  section = {
    titleformat+  = \label{sec\thesection}\raggedright
  }
}
\newcommand\chapternamebox[1]{%
\parbox{\ccwd}{\linespread{1}\selectfont\centering #1}}



\usepackage[inline]{enumitem}
\let\mod\relax
\DeclareMathOperator{\mod}{mod}

\DeclareMathOperator{\ee}{\!\!\;\mathrm e}
\newcommand{\MR}{\mathbb R}
\newcommand{\MQ}{\mathbb Q}
\newcommand{\MC}{\mathbb C}
\newcommand{\MF}{\mathbb F}
\newcommand{\MZ}{\mathbb Z}
\newcommand{\MN}{\mathbb N}
\newcommand{\MCF}{\mathscr F}

\newenvironment{eenum}{\begin{enumerate}
[label=\it\arabic*.\protect\label{\theexample.\arabic{enumi}},left=0.5cm,
ref=\it\arabic*]}
{
\end{enumerate}}
\newenvironment{enuma}
{\begin{enumerate*}
  [label=\it\arabic*.\protect\label{\theexample.\arabic{enumi}},
  itemjoin=\\\hspace*{0.5cm},ref=\it\arabic*]}
{
\end{enumerate*}}
\newenvironment{enumb}
{\begin{enumerate*}
  [label=\it\arabic*.\protect\label{\theexample.\arabic{enumi}},
  itemjoin=\\,ref=\it\arabic*]}
{
\end{enumerate*}}
\newenvironment{enumc}
{\begin{enumerate}
  [label=\it\arabic*.\protect\label{\theexample.\arabic{enumi}},
  left=0cm,ref=\it\arabic*]}
{
\end{enumerate}}

\newcommand\quan[1]{
\tikz[baseline=(a.base)]\node(a)[inner sep=0.5pt,draw,circle]{$#1$};
}
\newcommand\closure[1]{%
{}\mkern2mu\overline{\mkern-2mu#1}
}
\renewcommand\bar{\closure}



\newcounter{example}
\counterwithin{example}{section}
\renewcommand\theexample{\thechapter.\arabic{section}.\arabic{example}}
\newenvironment{example}{\par%
  \refstepcounter{example}%
  \hypertarget{ex\theexample}{}\label{ex\theexample}
  \noindent\hyperlink{ans\theexample}{\textbf{\theexample}}
}{\par}

\newcounter{ans}
\counterwithin{ans}{section}
\renewcommand\theans{\thechapter.\arabic{section}.\arabic{ans}}
\newenvironment{ans}{\par%
  \refstepcounter{ans}%
  \hypertarget{ans\theans}{}
  \noindent\hyperlink{ex\theans}{\textbf{\theans}}
}{\par}



%\usepackage[amsmath,thmmarks]{ntheorem}
%{
%\theoremstyle{nonumberplain}
%\theoremheaderfont{\bfseries}
%\theorembodyfont{\normalfont}
%\theoremsymbol{\mbox{$\Box$}}
%\newtheorem{proof}{\noindent 证\,}
%\newtheorem{solution}{\noindent 解\,}
%}
\usepackage{amsthm}
\newtheorem{lemma}{\noindent 引理\,}[section]

\setlist{nosep}



\newcommand\pp[2]{\frac{\partial #1}{\partial #2}}
\newcommand\ppp[2]{\frac{\partial^2 #1}{\partial #2^2}}
\newcommand\pppp[3]{\frac{\partial^2 #1}{\partial #2\partial #3}}
\newcommand\dd[2]{\frac{\mathrm d#1}{\mathrm d#2}}
\newcommand\ddd[2]{\frac{\mathrm d^2#1}{\mathrm d#2^2}}
\newcommand\dx{\mathrm dx}
\newcommand\dy{\mathrm dy}
\newcommand\dz{\mathrm dz}
\newcommand\dzz{\mathrm d\bar z}
\usepackage[subrefformat=parens]{subcaption}
\DeclareMathOperator{\Arg}{Arg}
\DeclareMathOperator{\vol}{vol}
%\usepackage{tkz-euclide}

\catcode`\;=\active
\newcommand{;}{\text{;}}
\numberwithin{equation}{section}
\DeclareMathOperator\diam{diam}
\DeclareMathOperator{\Log}{Log}
\DeclareMathOperator{\supp}{supp}
\DeclareMathOperator{\Aut}{Aut}
\DeclareMathOperator{\SL}{SL}
\DeclareMathOperator*{\Res}{Res}
\newcommand\DD{D}
\catcode`\;=13
\newcommand{;}{\text{;}}
\catcode`\:=13
\newcommand{:}{\text{:}}
\let\oldlim\lim
\def\lim{\oldlim\limits}
\let\oldsup\sup
\def\sup{\oldsup\limits}

\newcounter{method}
\counterwithin{method}{ans}
\renewcommand\themethod{方法\chinese{method}\hspace{0.5\ccwd}}
\def\method{\noindent\refstepcounter{method}\textbf{\themethod}}

\usepackage[toc]{multitoc}
\def\bd{\boldsymbol}
\def\VA{\bd A}
\def\VB{\bd B}
\def\VC{\bd C}
\def\VD{\bd D}
\def\VE{\bd E}
\def\VO{\bd O}
\def\VQ{\bd Q}
\def\VP{\bd P}
\def\VX{\bd X}
\def\ba{\bd \alpha}
\def\bb{\bd \beta}
\def\bg{\bd \gamma}
\def\vx{\bd x}
\def\vy{\bd y}
\def\zero{\bd 0}
\def\TT{^{\mathrm T}}
