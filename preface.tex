\chapter*{\kaishu 前言}
\addcontentsline{toc}{chapter}{前言}
  从1977年开始,加州大学伯克利数学系分校推行笔试考试,作为博士学位的一项重要的考核指标,这项考试替代了一些列标准的资格考试. 考试的目的是决定博士一年级的学生是否已经很好地掌握了基础数学知识,在博士生项目中有合适的机会取得成功.

  历史的原因,每次考试只有一半的考生可以通过,并且学生有三次考试机会. 从一开始,这项考试就是在博士学位获取过程中需要客服的重大障碍,因此,它也是在伯克利的博士项目中取得成功的一个最基本的衡量要求. 即便学生有三次机会,大多数人都觉得完成此要求最合适的时间是在第一个月,而不是在学期中或者学期末. 本书就是在这个前提下构思的,目的也是宣传考试的内容,并且帮助本科生在应对考试内容的时候做好准备.

  这项考试现在现在一年有两次,在每个学期的第二周进行. 在两天内,每天6个小时的时间
  来完成9个题目的笔试任务(在1998年以前是10题). 学生从9个题里面选6个(1988年以前是10选7). 大多数考试内容都包含了分析和代数,这是数学系本科生应该良好训练的部分. 

  以前伯克利在季度制度的时候,考试是一年三次:春,夏和秋. 从1986年起,考试就只有一年两次了,分别在一月和二月.

  从1981年秋季第一次考试起,制度是:允许考两次,每次6个小时,总共14/20题. 从1982年冬季到1988年春季,制度是:允许考两次,每次考试8个小时,共14/20题. 从1988年秋季看似,制度是:允许考三次,每次考试6个小时,共12/18题. 在所有的情形中,考试必须要在博士项目开始的13个月内通过.
  
  这本书汇编了超过1000道在过去几十年中出现在预备考试中的题,并且现在已经组成了一个非常值得学习的习题集,并且附上了相应的答案. 题目是按学科分类,在各科目内题目难度都是递增的. 当然有时候一个题目可能涉及好几个学科,分类并不容易. 有的题目从现在的角度来看,可能难度比较小,但是绝大部分题目都是相当具有参考价值的,且不少题已经被国内的优秀书籍所收录,相信本书对读者会大有裨益.

  本书用\LaTeX{}手打,作者翻译水平亦有限,个中错误在所难免,请读者不吝指正. 